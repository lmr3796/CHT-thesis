%\begin{IEEEkeywords}
%
%\end{IEEEkeywords}

\begin{abstractzh}
  越來越多的企業與機構開始建造自己的資料中心作為私有雲(private
  cloud)使用。在這些資料中心裡,機器的性質與能力會因為新採購與
  被淘汰而有所不同,但由於採購通常只會在年度或半年度時進行,在這
  中間,機器的數量是固定的。
  在這樣的異質性環境下,根據不同工作(job)的資源需求與特性去進
  行排程,使得它們可以盡量在其要求的時間限制前完成是很重要的。

  在本論文中,我們提出了一個雲端資源管理框架以動態的調配在系統
  中各個工作所能夠使用的運算節點(node)數量,進而達到有效率整
  個系統的運算資源。

  關鍵字:工作排程、異質性伺服器叢集、資源分配
\end{abstractzh}

\begin{abstracten}
  Many enterprises or institutes are building private clouds by
  establishing their own data center.
  In such data centers, the physical machines can be different due to
  annual upgrades,  but the amount of machine are fixed for most of the
  time.
  In such heterogeneous environment, scheduling jobs with different
  resource requirements and characteristic in order to meet different
  timing constraints is important.

  In this paper, we proposed a cloud resource management framework to
  dynamically adjust the number of computation nodes for every job in
  the system.

  keywords: Job/Task Scheduling, Heterogeneous Clusters, Resource Allocation
\end{abstracten}

%\begin{comment}
%\terms{}
%\keywords{Job/Task scheduling; Heterogeneous Clusters; Resource allocation; }
%\end{comment}
