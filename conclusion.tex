\chapter{Conclusion}\label{chap:conclusion}

Many enterprises or institutes are building private clouds by
establishing their own data center.
In such data centers, the physical machines can be different due to 
annual upgrades,  but the amount of machine are fixed for most of the 
time.
In such heterogeneous environment, scheduling jobs with different 
resource requirements and characteristic in order to meet different 
timing constraints is important.

In this paper, we proposed a cloud resource management framework to
dynamically adjust the number of computation nodes for every job in the
system.
This framework make decisions according to some specified policies.
The proposed framework can work as an individual cloud computing system,
or as extension components of an existent cloud system.
Our experiment results demonstrated that our system is capable of
dynamically adjusting the resource allocation plan according to the
runtime statistics to meet the deadline requirements even there is
worker failure.
Moreover, leveraging this ability to dynamically adjust resource usage,
our system can adopt more aggressive scheduling policies that do not
need to preserve resources in advance and thus improves its throughput
and performance.

Aside from the framework, we also proposed several scheduling policies
for different situations, including 2 heterogeneity-aware ones.
The deadline-based policy is a heterogeneity-aware policy that takes
both priority and deadline into consideration.
Our experiment shows that
\begin{enumerate}
  \item Earliest Deadline First algorithm, an optimal solution in
    real-time systems, still gives the best performance.
  \item Blending the deadline-based policy with EDF by allocating the
    spare resource to jobs with earliest deadlines improves its
    performance and almost make the performance reach the level of EDF.
\end{enumerate}
Although still not outperforming EDF, deadline-based policy with spare
resource filled by deadline gives a nice balance between priority and
deadline constraints.
It takes priority into consideration and give high priority jobs enough
resource first.
