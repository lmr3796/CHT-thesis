\chapter{Related Work}\label{sec:related}

Job scheduling with deadlines has been a well studied field.
In real-time systems, where each job has a hard deadline to be done,
Earliest Deadline First (EDF) policy has been widely adapted.
It is an optimal dynamic schedule policy on preemptive
uniprocessors~\cite{cite:pinedo2012scheduling}, which means if all
submit jobs in the system can be scheduled in a way that every job can
be done before deadline, EDF is able to schedule them so that all of
them can finish before their deadlines.

Various works studied dynamic scheduling methods for parallel real-time
jobs (PRJs) in clusters.
Qin et al. modeled parallel real-time jobs as directed acyclic graphs
and proposed a reliability-driven
method~\cite{cite:qin-reliability-driven}.
He et al. represented real-time processing as hybrid execution of
existing periodic jobs and newly arriving aperiodic jobs, and proposed a
scheduling method by modeling spare capabilities of new arriving
aperiodic parallel real-time jobs~\cite{cite:he-spare-capabilities}.
Xie et al. proposed TAPADS, Task Allocation for Parallel Applications
with Deadline and Security Constraints, which takes security constraints
into consideration~\cite{cite:xie-TAPADS,cite:xie2008security}.
In these cases, deadlines are hard.
If the system cannot guarantee a submitted job can be finished before
required deadline, it is thus rejected.

