\chapter{Scheduling Policies}\label{chap:policy}

In this chapter, we introduce the four policies in our system ---
\emph{priority-based}, \emph{proportion-based}, \emph{workload-based}
and \emph{deadline-based}.
Each policy requires different parameters such as job priority,
execution efficiency per different server and execution deadline.
\emph{Decision maker} makes dispatching plans to according to a
specified policy.

When designing the schedule policies, we should take the number of tasks
in a job into consideration, since we don't want to assign workers more
than tasks of a job to it.
Recall that it is the end users' responsibility to split a job into well
balanced tasks; the more balanced the tasks, the better the system
schedules.


\section{Priority-based Policy}

%%% Introduce Priority-based
Priority-based policy distributes workloads according to job priority. 
We first sort all jobs in the system according to their priority in 
descending order.
Next, starting from the job with the highest priority, we assign $x$
workers to the job, where $x$ is the maximum of 1) number of idle worker
and 2) number of tasks in the job.
The process is repeated until there are no idle workers or all the jobs
are assigned with sufficient workers.
The idea behind this policy is to make the job with highest priority 
runs as fast as possible.
However, this assignment might cause \emph{starvation} to jobs with 
low priority.

%%% How we avoid starvation from happening
To avoid \emph{starvation}, we preserve a portion for low priority jobs.
We set a reservation rate $r \in [0,1]$ that the portion $r$ of workers
will be reserved for low priority jobs.
The job with highest priorities can use at most the $1-r$ of all workers
and the remaining workers, including those are not reserved but still not
assigned to the highest priority job because of the task amount limit,
will be assigned to jobs with lower priority, in the manner that one job
receives one worker.
The complete algorithm is shown as algorithm~\ref{algo:priority-based}.

\begin{algorithm}[htbp]
  \KwIn{
  $workerSet=\{w_1, w_2, \ldots, w_m\}$,
  $jobSet=\{j_1, j_2, \ldots, j_n\}$,
  preserving rate $r\in[0,1]$
}
\KwOut{A mapping of each job to scheduled workers}
$result \gets$
KeyValueMap(key $\to \{\}$ for all key $\in jobSet$)\;
$jobSet \gets$ $sortByPriority(jobSet)$\;
$c \gets$ $min(jobSet[0]$.totalTask, $workerSet$.size$-floor((1-r) *
workerSet.$size$)$\;
$result[jobSet[0]] \gets$ $\{w_1, ..., w_c\}$\;
\For{$i=1$ to $min(m-c, n)$}{
  $result[jobSet[i]]$ $\gets$ $\{w_{c+i}\}$\;
}
\Return{result}\;


  \caption{Priority-based policy}
  \label{algo:priority-based}
\end{algorithm}

\section{Proportion-based Policy}

The main concern of the proportion-based policy is the workload
proportion of a job to all the others.
In contrast to priority-based policy, this policy does not take job
priorities into consideration unless the priority of a task is strongly
related to its workload.
It first sums all the workloads from each job, then compute the workload
ratio of the individual job.
The number of workers a job is assigned equals to the number of all 
workers multiplies it ratio.
This algorithm, as shown in algorithm~\ref{algo:proportion-based}, is
also known as \emph{fair share
scheduling}~\cite{cite:fair-share-scheduling} that workers a job should
be allocated is in proportion to its workload proportion to all jobs. 

Proportion-based is a suitable solution for streaming jobs or jobs with
dependencies.
For streaming jobs with vary input rate, the number of workers needed is
highly related to its input rate, thus workload.
As for jobs with dependencies, a job may need the output from other 
jobs as input.
Therefore, we should assign more workers to the predecessor in the
beginning, and reduce the number of workers according to time.
On the other hand, the number of worker of a successor will be 
increasing since the workload from the predecessor increased.

\begin{algorithm}[htbp]
  \KwIn{
  $workerSet=\{w_1, w_2, \ldots, w_m\}$,
  $jobSet=\{j_1, j_2, \ldots, j_n\}$,
}
\KwOut{A mapping of each job to scheduled workers}
$result \gets$ KeyValueMap(key $\to \{\}$ for all key $\in$
$jobSet$)\;
$jobSet \gets$ $sortByPriority(jobSet)$\;
\For{$i=1$ to $m$}{
  break if $workerSet$.size equals to 0 \;
  $c \gets$
  $min(ceil(\frac{jobSet[i].\text{workload}}{\text{total
  workload}})), workerSet.\text{size})$ \;
  $result[jobSet[i]] \gets \{\text{first $c$ element of
  $workerSet$}\}$\;
  $workerSet \setminus result[jobSet[i]]$\;
}
\Return{result}\;

  \caption{Proportion-based policy}
  \label{algo:proportion-based}
\end{algorithm}


\section{Workload-based Policy}

%%% [DS] The only different is worker heterogeneity?
%%% What is "workload requirement of $J_m$"

Similar to the \emph{proportion-based} policy, the \emph{workload-based}
policy takes the workload of jobs as scheduling metric.
However, workload-based takes worker heterogeneity, i.e. each
worker may have different computing ability, into consideration.
This policy tries to finish every job by assigning jobs to the most
suitable workers, i.e. workers with higher throughput.

\subsection{Model Definition}

We represent each job $j_i$ submitted as $j_i = (w_i, p_i, n_i)$, where
$w_i$ is the estimated workload, $p_i$ is the job priority, and $n_i$
is the number of tasks in $j_i$.
As for the workers, we model each worker $s_k$ as a vector of estimated
\emph{throughput} of each submitted job; we can write is as $s_k =
(th^k_1, th^k_2, \ldots)$.

\subsection{Algorithm}

We apply a greedy approach in this policy.
First, we sort all the jobs according to their priority in descending
order.
Starts from the job with the highest priority $J_m$, sort the remaining
workers according to their throughput $th_m$ to $J_m$.
Then assign the top $k$ workers that can satisfy the workload
requirement of $J_m$, and remove assigned workers from the list.
The job is also removed from the list waiting for assignment.
Repeat this process until there are no idle worker or unassigned job.
The complete algorithm is shown as algorithm ~\ref{algo:workload-based}.

\begin{algorithm}[htbp]
\KwIn{
  $workerSet=\{w_1, w_2, \ldots, w_m\}$,
  $jobSet=\{j_1, j_2, \ldots, j_n\}$,
}
\KwOut{A mapping of each job to scheduled workers}
$result \gets$
KeyValueMap(key $\to \{\}$ for all key $\in jobSet$)\;
$jobSet \gets$ $sortByPriority(jobSet)$\;
\For{each job $j \in jobSet$}{
  \If{$workerSet$ is empty}{\textbf{break}\;}
  $cmp \gets function(w)$\{\Return $w$.throughput$[j]\}$\;
  $workerSet \gets$ $sort(workerSet,cmp,ACSENDING)$\;
  $i \gets 0$\;
  $throughput \gets 0$\;
  %\For{$i = 0 $ to $workerSet$.size-1}{
  %  $throughput \mathrel{+}= workerSet[i].throughput[j]$\;
  %  \If{$throughput \geq j$.workload}{
  %    \textbf{break}\;
  %  }
  %}
  \For{$i = 0$ ;\\
    $i<workerSet$.size \textbf{AND}
    $throughput < j$.workload;
    $i\mathrel{+}\mathrel{+}$
  }{
    $throughput \mathrel{+}= workerSet[i].throughput[j]$\;
  }
  $result[j] \gets workerSet[0...i]$\;
  $workerSet$.remove(0,$i$)
}

\Return{result}\;

\caption{Workload-based policy}
\label{algo:workload-based}
\end{algorithm}


\section{Deadline-based Policy}

\emph{Deadline-based} policy tries to finish every jobs before their 
soft deadline.
If a feasible scheduling plan does not exist, this policy tends to meet
the deadlines for jobs with higher priority.
There are varies kinds of jobs that must be finished before a given 
time constraint.
For example, as a Internet service provider, settling monthly bills of
millions of users within few days after the monthly charge-off day is
very critical to their business.
It it obvious that computing resource allocated to this kind of job 
should increase as their deadline approaches --- the closer the 
deadline is, the more worker a job should get.
Unfortunately, policies introduced above does not directly adapt to 
this need; hence, we introduce this policy designed for
\emph{deadline-aware} scenarios.

\subsection{Model Definition}

For this policy, we assume that each job is provided with priority and
deadline.
The execution time for each job is assumed to be known, which can be 
estimated by historical data or profiling.
A job $j_i$ can thus be represented as $j_i = (d_i, p_i, n_i)$, where
$d_i, p_i, n_i$ refers to the job deadline, job priority, and the 
number of tasks in $j_i$, respectively
The workers are modeled as a vector of estimated execution time of each 
submitted job: $s_k = (T^k_1, T^k_2, \ldots)$, similar to the previous 
policy.

\subsection{Algorithm}

A job should be finished before its soft deadline.
Intuitively, for the same job, if the deadline is twice tighter, it
requires twice as many workers to meet that deadline.
More generally speaking, the deadline and the workload are in reciprocal
relationship.
This implies that an worker allocation $S$ to job $j_i$ meeting the
deadline is equivalent to
\[\displaystyle\sum_{s_k \in S}\frac{1}{T^k_{j_i}} \geq \frac{1}{d_i}\]
With this observation, only a little modification on the
\emph{workload-based} policy is needed to apply to this model: simply
use the inverse of deadline as throughput.

First, sort all the jobs according to their priority in descending
order.
Starting from the job with the highest priority $J_m$, sort the
remaining workers according to their throughput $th_m$ to $J_m$.
Then assign the top $k$ workers that can finish $J_m$ before it 
deadline, and remove assigned workers from the list.
The job is also removed from the list waiting for assignment.
Repeat this process until there are no idle worker or un-assigned job.
The complete algorithm is shown as algorithm ~\ref{algo:deadline-based}.

\begin{algorithm}[htbp]
  \KwIn{
  $workerSet=\{w_1, w_2, \ldots, w_m\}$,
  $jobSet=\{j_1, j_2, \ldots, j_n\}$,
}
\KwOut{A mapping of each job to scheduled workers}
$result \gets$
KeyValueMap(key $\to \{\}$ for all key $\in jobSet$)\;
$jobSet \gets$ $sortByPriority(jobSet)$\;
\For{each job $j_i=(d_i,p_i,n_i)  \in jobSet$}{
  \If{$workerSet$ is empty}{\textbf{break}\;}
  $cmp \gets function(w)$\{\Return $1/w.execTime[j_i]\}$\;
  $workerSet \gets$ $sort(workerSet,cmp,DECSENDING)$\;
  $th \gets 0$\;
  \For{\textbf{(} $k = 0$ ;\\
    $k<workerSet$.size \textbf{AND}
    $k<n_i$ \textbf{AND}
    $th < 1/w_i.execTime[j_i]$;
    $k \gets k+1 $\textbf{)}
  }{
    $th \mathrel{+}= 1/workerSet[k].execTime[j_i]$\;
  }
  $result[j_i] \gets workerSet[0...i]$\;
  $workerSet$.remove(0,$k$)
}
assign remaining workers to jobs still in need\;
\Return{result}\;

  \caption{Deadline-based policy}
  \label{algo:deadline-based}
\end{algorithm}

\section{Spare Resource Allocating}

Originally, scheduling policies should consider preserving resource for
possible emergency jobs like workload-based or deadline-based policy.
With the adaptive adjustment in section\ref{sec:adaptive}, we can
aggressively allocate the spare resource as much as possible.
There are 2 ways to do so, allocate them by priority or by closest
deadline.
For jobs still requires worker to process, sort them by the
criteria (priority or deadline), and allocate the rest workers one by
one until workers are used up or every job has obtained enough workers.
This is the additional optimization we did on the policies that
preserves resource in advance for emergency jobs like workload-based or
deadline-based policy.
