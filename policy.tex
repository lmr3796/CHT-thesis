\chapter{Policy}

Here we are going to introduce three dispatching policies --
\emph{priority-based}, \emph{proportion-based} and
\emph{deadline-based}.
Each of them requires different parameters such as job priority,
execution efficiency per different server and execution deadline.
Users can specify a feasible policy for their need.
The decision maker will make dispatching plans to according to the
specified policy.


\section{Priority-based Policy}

The idea behind this policy is to make the job with highest priority
runs as fast as possible.
Intuitively, we can sort the jobs according to their priority then
greedily schedule as many workers as possible to each job in that order.
However, this might cause \emph{starvation} that lower priority jobs
never get executed.
To solve this, we preserve a portion $r \in [0,1]$ of workers for low
priority jobs and job with the highest priority can use only the
unpreserved workers.
The algorithm is shown as algorithm~\ref{algo:priority-based}.

\begin{algorithm}[H]
  \DontPrintSemicolon % Some LaTeX compilers require you to use
  \KwIn{
    $workerSet=\{w_1, w_2, \ldots, w_m\}$,
    $jobSet=\{j_1, j_2, \ldots, j_n\}$,
    preserving rate $r\in[0,1]$
  }
  \KwOut{A mapping of each job to scheduled workers}
  $result \gets$
  KeyValueMap(key $\to \{\}$ for all key $\in jobSet$)\;
  $jobSet \gets$ $sortByPriority(jobSet)$\;
  $c \gets$ $min(jobSet[0]$.totalTask, $workerSet$.size$-floor((1-r) *
      workerSet.$size$)$\;
  $result[jobSet[0]] \gets$ $\{w_1, ..., w_c\}$\;
  \For{$i=1$ to $min(m-c, n)$}{
    $result[jobSet[i]]$ $\gets$ $\{w_{c+i}\}$\;
  }
  \Return{result}\;
  \caption{Priority-based policy}
  \label{algo:priority-based}
\end{algorithm}

\section{Proportion-based Policy}

In contrast to priority-based, this policy is takes job priorities into
consideration very little; instead, the main concern of this policy is
the workload proportion of a job to all the others.
This algorithm, as shown in algorithm~\ref{algo:proportion-based}, is
also known as fair share scheduling: workers a job should be allocated
is in proportion to its workload proportion to all jobs. 

\begin{algorithm}[H]
  \DontPrintSemicolon % Some LaTeX compilers require you to use
  \KwIn{
    $workerSet=\{w_1, w_2, \ldots, w_m\}$,
    $jobSet=\{j_1, j_2, \ldots, j_n\}$,
  }
  $result \gets$ KeyValueMap(key $\to \{\}$ for all key $\in$
  $jobSet$)\;
  $jobSet \gets$ $sortByPriority(jobSet)$\;
  \For{$i=1$ to $m$}{
    break if $workerSet$.size equals to 0 \;
    $c \gets$
    $min(ceil(\frac{jobSet[i].\text{workload}}{\text{total
        workload}})), workerSet.\text{size})$ \;
    $result[jobSet[i]] \gets \{\text{first $c$ element of
    $workerSet$}\}$\;
    $workerSet \setminus result[jobSet[i]]$\;
  }
  \Return{result}\;
  \caption{Proportion-based policy}
  \label{algo:proportion-based}
\end{algorithm}

\section{Deadline-based Policy}

Plenty kinds of jobs are required to be finished within a given
deadline.
For example, as a Internet service provider, settling monthly bills of
millions of users within few days after the monthly charge-off day is
very critical to their business.
\emph{Computing resource allocated this kind of job should increase as
their deadline approaches}.
However, policies introduced in previous sections can't adapt to this
need; hence, we introduce another policy designed for
\emph{deadline-aware} scenarios.

\subsection{Model Definition}


\begin{algorithm}[H]
  \DontPrintSemicolon % Some LaTeX compilers require you to use
  \KwIn{
    $workerSet=\{w_1, w_2, \ldots, w_m\}$,
    $jobSet=\{j_1, j_2, \ldots, j_n\}$,
  }
  % TODO Implement the algorithm

  \Return{result}\;
  \caption{Deadline-based policy}
  \label{algo:deadline-based}
\end{algorithm}
