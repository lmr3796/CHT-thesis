\chapter{System Architecture}

\begin{figure}
  \centering
  \input{figures/major-comp-rela}
  \caption{System Architecture Overview}
  \label{fig:archi-overview}
\end{figure}

Figure~\ref{fig:archi-overview} shows the architecture of this cloud
management system, which consists of three components: the \emph{status
checker}, the \emph{decision maker} and the \emph{dispatcher}.  We
implemented each of the components as a separate RPC (remote procedural
call) server.  Separate RPC server implementation makes each component
pluggable, which means


%so that the system can be

%We'll discuss each component in the following sections.




The status checker periodically collects the information about physical
servers such as whether a node is down, busy or idle and its resource
usage.  Such information is provided to the decision maker as reference
for making allocation adjustment.  The decision maker is in charge of
making and adjusting resource allocation plans according to the
specified policy that takes job deadline, priority, etc. into
consideration; it is designed to be in passive mode, which means it is
invoked (by the dispatcher) only under certain circumstances (which will
be discussed later) rather than periodically.  The dispatcher is the
component that deal with the physical resource allocation adjustment
according to the allocation plan made by the decision maker.

\section{Status Checker}

\section{Decision Maker}

\section{Dispatcher}

\section{Work Flow}

