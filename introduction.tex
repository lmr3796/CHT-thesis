\chapter{Introduction}\label{chap:intro}

%%% TODO Starts with introduction of cloud computing
Cloud computing has become a popular issue in recent years.
Many enterprises or institutes are building private clouds by
establishing their own data centers.
In such data centers, the maximum amount of computing resources and the
number of physical machines are fixed for most of the time.
They change only during adding newly purchased servers or removal of
obsolete ones during equipment upgrade.

%%% TODO why is job deployment so important

Jobs in a data center may vary from one another.
They may have different characteristic, resource requirements, time
constraints, and priority.
For example, scientific computation requires large computing resources,
while a billing sub-system need to generate the credit-card bills for
each user every month.
Also, newly arrival emergency jobs may require large amount of resource
in a short period.
How to allocate resources for jobs in a heterogeneous environment and
meet different requirements becomes an important issue.

In this paper, we proposed a cloud resource management framework to
adjust the number of computation nodes for every job in the system
dynamically.
This framework makes decisions according to some specified policies ---
\emph{priority-based}, \emph{proportion-based}, \emph{workload-based}
and \emph{deadline-based}.
These policies take the remaining workload, priority or deadline of each
job into consideration and generate a scheduling plan.
Administrators can choose a different policy or specify a customized one
for the system in order to fit the their needs.
Moreover, this framework can work as an individual cluster managment
system or as extension components of an existent cloud system.

The rest of the paper is organized as followed.
The system architecture is presented in Chapter~\ref{chap:arch} and the
implementation is discussed in Chapter~\ref{chap:arch}.
Chapter~\ref{chap:policy} introduce the four policies we implement for
different types of jobs.
The experimental results are in Chapter~\ref{chap:exp}.
Chapter~\ref{chap:conclusion} is the conclusion.
