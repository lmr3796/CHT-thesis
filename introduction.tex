\section{Introduction}\label{sec:intro}

%%% TODO Starts with introduction of cloud computing
%%% TODO Add one or two paragraphs about cloud computing before going to scalability
%%% TODO Something like "cloud computing has become a popular issue in recent year" etc.

%%% Followed by scalability
Scalability is one of the key characteristics of cloud computing.
%%% Define/explain scalability and its importance before the example
% TODO Add example why this is the key???? Why is it important?

% TODO Comment out virtualization...
% TODO But must replace with workloads in cloud computing/data center

%%% How to achieve scalability: virtualization
Virtualization wraps essential software components into virtual machines
in order to make cloud systems scalable.
For example, Lombardi and Di Pietro proposed ACPS (Advanced Cloud
Protection System) that wraps applications, operating systems and
libraries into virtual machines, and deploy them on physical servers for
execution~\cite{cite:secure_virt_for_cloud, cite:cloud_issue}.
After we wrap essential software components into virtual machines, we
can deploy them dynamically.
Consequently we can quickly adjust the number of running virtual
machines to satisfy user demands.

%%% TODO why is job deployment so important


%%% TODO Rewrite virtual machines into jobs
%%% Gap: too many VMs on a single server cause resource competition
Properly placing multiple virtual machines on a single physical machine
will utilize its hardware resource more efficiently.
As a hypervisor deploys more virtual machines on a physical machine, and
allocates resources for them, it improves system utilization by reducing
the possibility of resource sitting idle.
However, as the number of virtual machines in a physical machine
increases, the resource each virtual machine receives will
decrease~\cite{cite:resource_overbooking}.
For example, if we deploy a single virtual machines on a physical
machine, it can utilize all the physical resources.
However, if we deploy four virtual machines on the same physical
machine, each of them will receive a quarter of the hardware resource,
which makes applications running on these virtual machines perform
worse.

% TODO Fuck off those VMs, change it into a "for example"
%%% One solution: dedicated VM
Running applications on dedicated servers can guarantee performance.
A dedicated server~\cite{cite:dedicated_hosting} runs at most one
virtual machine at a time, so that the performance of the virtual
machine will not be interfered by other virtual machines.
%TODO : More dedicate server and performance guarantee examples here.
%CHT examples here.

%%% TODO Emphasize 'heterogeneous'
The concept of dedicated server is feasible in data centers.
The number of physical servers of a data center is usually
fixed~\cite{cite:maintenance_framework}, and it only changes
infrequently when we add newly purchased servers, or remove obsolete
ones, during equipment upgrade.
In other words, the number of physical servers within a data center is
fixed between upgrades.
Since the server configuration is fixed, we can statically map
applications to dedicated servers.

%%% describe target environment: CHT applications/workloads
Under the assumption that the number of physical servers in a data
center is fixed, determining how many dedicated servers should be
allocated for each job or application becomes a critical problem:

Matching application instances to dedicated servers is crucial to
performance.
% TODO  examples

% TODO  Paragraph for bare-bone job dispatch

%%% The goal of this work: develop a system that dynamically distributes workloads to physical servers according to some policies.

% TODO Merge this sentence into the CHT application description
% Not revised yet!!!
Jobs with different priority requiring different processing throughput
will start and finish at different time; handling these varying
conditions and dynamically adjusting allocated resource of each job so
that every job could meet its requirement is what we want to do.

The goal of this project is to develop a cloud resource management
system that dynamically adjusts the number of dedicated server a single
job is granted to use according to a specified policy with regard to the
remaining workload, priority and deadline of the job.
The data center manager can choose a different policy to fit the their
need or specify a customized policy for the system; moreover, this
management system shall be able to cooperate with existing cloud
operating systems.

% There's a remaining paragraph in Simon's proposal which is added per
% CHT's request, but I think it's not appropriate to put it in the
% background section.
