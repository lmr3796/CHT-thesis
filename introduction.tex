\section{Introduction}\label{sec:intro}

%%% TODO Starts with introduction of cloud computing
Cloud computing has become a popular issue in recent year.
Many enterprises or institutes are building private clouds by 
establishing their own data center.
In such data centers, the maximum amount of computing resources, or 
number of physical machines, are fixed for most of the time.
It only changes during adding newly purchased servers, or remove 
obsolete ones during equipment upgrade.

%%% TODO why is job deployment so important

Jobs in a data center vary from one another.
Jobs may have different characteristic, resource requirements, time 
constraints, and priority.
For example, scientific computation requires large computing resources,
while a billing sub-system need to generate the credit-card bill for 
each user every month.
Also, newly arrival emergency jobs may require large amount of resource
in a short time.
How to allocate jobs in a heterogeneous environment in order to meet 
different requirements becomes an important issue.

In this paper, we proposed a cloud resource management framework to 
dynamically adjust the number of computation nodes for every job in the
system.
This framework make decisions according to some specified policies --
\emph{priority-based}, \emph{proportion-based}, \emph{workload-based}
and \emph{deadline-based}.
The policies take the remaining workload, priority or deadline of each
job into consideration, and generate a scheduling plan that satisfies 
the job requirements.
This framework can work as an individual cloud computing system, or as
extension components of an existent cloud system.

The rest of the paper is organized as followed.
The system architecture is presented in Section~\ref{sec:arch}.
Section~\ref{sec:policy} introduce the four policies we implement for 
different types of jobs.
The experimental results are in Section~\ref{sec:exp}.
Section~\ref{sec:conclusion} is the conclusion.
