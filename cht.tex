\documentclass{beamer}

%%% Compile flags
% Show algorithm pseudocode
\newif\ifshowalgo
\showalgofalse

% Put required libraries into library.tex
\usepackage[linesnumbered,ruled]{algorithm2e}
\usepackage{amsmath}
\usepackage{float}
\usepackage{listings}
\usepackage{url}
\usepackage{setspace}
\usepackage{tikz}

\usetikzlibrary{arrows}
\usetikzlibrary[backgrounds]
\usetikzlibrary{calc}
\usetikzlibrary{decorations.markings}
\usetikzlibrary{fit}
\usetikzlibrary{shadows}
\usetikzlibrary{shapes.arrows}



\usepackage{etoolbox}
\usepackage{graphicx}
\usepackage{xeCJK}
\setCJKmainfont{微軟正黑體} 

% Tikz figure styls
% System architecture
\tikzstyle{vecArrow} = [thick, decoration={markings,mark=at position
  1 with {\arrow[semithick]{open triangle 60}}},
  double distance=1.4pt, shorten >= 5.5pt, preaction = {decorate},
  postaction = {draw,line width=2pt, white,shorten >= 4.5pt}
]
\tikzstyle{component} = [draw,text centered,rounded corners,drop
  shadow,text width=8em,fill=red!20,minimum height=6em
]
\tikzstyle{dispatcher} = [component,minimum height=19.3em]
\tikzstyle{system}=[draw,dashed,rounded corners,fill=yellow!30]

% Job allocations
\tikzstyle{worker-matrix}=[
  row sep=0.7cm,column sep=0.7cm,
  nodes={draw, minimum size=3em, circle, thick, fill=gray!20}
]
\tikzstyle{job-allocation}=[draw, thick, rectangle,rounded corners, inner sep=0.2cm,]
\tikzstyle{job-1}=[job-allocation, fill=green!15, label=left:$J_1$]
\tikzstyle{job-2}=[job-allocation, fill=blue!15, label=left:$J_2$]
\tikzstyle{job-3}=[job-allocation, fill=red!15, label=right:$J_3$]
\tikzstyle{invisible}=[draw=none, fill=none]


\usetheme{Boadilla}

\title{大型資料中心之硬體資源分配管理}
%\author{Ting-Chou Lin}

\AtBeginSection[]{
  \begin{frame}<beamer>
    \frametitle{Outline}
    \tableofcontents[
      currentsubsection,
      sectionstyle=show/shaded,
      subsectionstyle=show/shaded,
    ]
  \end{frame}
}

\begin{document}

\begin{frame}
  \titlepage
  \label{title-page}
\end{frame}
\begin{frame}
  \frametitle{Outline}
  \tableofcontents[
    currentsubsection,
    sectionstyle=show/show,
    subsectionstyle=show/shaded,
  ]
\end{frame}

\section{Introduction}

\begin{frame}
  \frametitle{Background}
  \setbeamercovered{transparent}
  \begin{itemize}[<+->]
    \item As the amount of data and computing demand increases, more and
      more companies tend to build their own data centers --- Private
      Cloud.
    \item For elasticity, rather than dedicating servers for particular
      applications, more people tend to build a large cluster for all
      needs.
  \end{itemize}
\end{frame}
\begin{frame}
  \frametitle{Background --- Job Characteristics}
  \setbeamercovered{transparent}
  \begin{itemize}[<+->]
    \item In data centers, jobs come with different characteristics.
      \begin{itemize}
        \item Importance/Priority
        \item Computation resource demand
          \begin{itemize}
            \item CPU bound versus I/O bound
            \item <.-> Requires large RAM
            \item <.-> Leverage GPGPU
          \end{itemize}
        \item <.(2)->{\alert<.(2)->{Deadline}}
      \end{itemize}
    \item Moreover, some of them should be completed before time
      constraints.
      \begin{itemize}
        \item <.-> Billing jobs should be done within days after
          charge-off.
        \item <.-> Research experiments are not that urgent.
      \end{itemize}
  \end{itemize}
\end{frame}
\begin{frame}
  \frametitle{Background --- Server Characteristics}
  \setbeamercovered{transparent}
  \begin{itemize}[<+->]
    \item Resources in data centers are usually \emph{limited} and
      \emph{fixed}
      \begin{itemize}
        \item Purchase new ones and remove obsolete ones only on
          periodic equipment upgrades.
        \item \textbf{The number of workers is fixed between upgrades.}
      \end{itemize}
    \item <.(-2)-> Servers in the cluster are \alert{heterogeneous}.
      \begin{itemize}
        \item Old servers coexist with new servers.
        \item <.-> Only few servers have special capabilities, like Giga-LAN
          or GPGPU.
      \end{itemize}
  \end{itemize}
\end{frame}
\begin{frame}
  \frametitle{Goal}
  \begin{itemize}
    \item Implement a resource management component that \emph{dynamically}
      adjusts resource for each job.
    \item System administrator can easily specify different policies to
      adapt to their needs.
    \item The system should take server \alert{heterogeneity} into consideration.
  \end{itemize}
\end{frame}
\begin{frame}
  \frametitle{Target Environment --- CHT's Data Center}
  \begin{itemize}
    \item Old and new servers coexists --- heterogeneous environment.
    \item Most jobs process files (billing accounting).
    \item Jobs are split into many \emph{tasks}, each can be processed
      in parallel.
    \item Data is shared via NFS.
    \item Servers may fail.
  \end{itemize}
\end{frame}


\section{System Overview}
\begin{frame}
  \frametitle{System Architecture -- Overview}
  \begin{columns}
    \begin{column}{.5\textwidth}
      \begin{figure}
        \resizebox{\linewidth}{!}{
          \pgfdeclarelayer{background}
\pgfdeclarelayer{foreground}
\pgfsetlayers{background,main,foreground}

\begin{tikzpicture}
  \node[](management){\pgfdeclarelayer{background}
\begin{tikzpicture}
  \node[](key-comp){\input{figures/manage-compo-relation.tikz}};
  \path (key-comp.south) + (0, -0.5) node(comp-caption){Management System};
  \begin{pgfonlayer}{background}
    \node[system,fit=(key-comp) (comp-caption)]{} ;
  \end{pgfonlayer}
\end{tikzpicture}
};

  % Client
  %\coordinate (client-mid) at (management.east) + (5em,0);
  %\path (client-mid) + (0,-2) node[component,text width=5em]{Client};
  %\path (client-mid) + (0, 2) node[component,text width=5em]{Client};

  % Workers
  \coordinate (l1) at (management.south west);
  \coordinate (l2) at (management.south east);
  \path (l1) + (0, -3)
  node[component,text width=4em,minimum height=3em](w1){worker};
  \path (l2) + (0, -3)
  node[component,text width=4em,minimum height=3em](w4){worker};
  \path ($(w1)!0.33!(w4)$)
  node[component,text width=4em,minimum height=3em](w2){worker};
  \path ($(w1)!0.66!(w4)$)
  node[component,text width=4em,minimum height=3em](w3){worker};
  \draw[vecArrow] ([xshift=-5]$(l1)!0.2!(l2)$) to ([xshift=-5,yshift=3]w1.north);
  \draw[vecArrow] ([xshift=-5]$(l1)!0.4!(l2)$) to ([xshift=-5,yshift=3]w2.north);
  \draw[vecArrow] ([xshift=-5]$(l1)!0.6!(l2)$) to ([xshift=-5,yshift=3]w3.north);
  \draw[vecArrow] ([xshift=-5]$(l1)!0.8!(l2)$) to ([xshift=-5,yshift=3]w4.north);
  \draw[vecArrow] ([xshift=5,yshift=3]w1.north) to ([xshift=5]$(l1)!0.2!(l2)$);
  \draw[vecArrow] ([xshift=5,yshift=3]w2.north) to ([xshift=5]$(l1)!0.4!(l2)$);
  \draw[vecArrow] ([xshift=5,yshift=3]w3.north) to ([xshift=5]$(l1)!0.6!(l2)$);
  \draw[vecArrow] ([xshift=5,yshift=3]w4.north) to ([xshift=5]$(l1)!0.8!(l2)$);
\end{tikzpicture}


        }
        \caption{Architecture Overview}
        \label{fig:archi-overview}
      \end{figure}
    \end{column}
    \begin{column}{.5\textwidth}
      \begin{itemize}
        \item Each component is a RPC server.
        \item As a cluster management system or extension
          components of existent cloud systems
      \end{itemize}
    \end{column}
  \end{columns}
\end{frame}

\section{Policy}

\begin{frame}
  \frametitle{Policy}
  \begin{itemize}
    \item The scheduling algorithm used by the system.
    \item Each of them requires different parameters.
      \begin{itemize}
        \item User provided
        \item System statistics
      \end{itemize}
    \item Administrators can select different policies to fit their
      need.
  \end{itemize}
\end{frame}
\begin{frame}
  \frametitle{Outline}
  \tableofcontents[
    currentsubsection,
    sectionstyle=show/shaded,
    subsectionstyle=show/show/shaded,
  ]
\end{frame}
\subsection{Priority-based}

\begin{frame}
  \frametitle{Policy -- Priority-based}
  %TODO for different needs, use different policies
  %If no deadline, only priority is specified. e.g., research experiment
  \only<1>{
    \begin{itemize}
      \item Finish the job with highest priority as fast as possible
      \item Preserve workers to relieve starvation of low priority ones
      \item Only priority is considered
      \item Good for those facility whose applications do not have 
        hard time limit, such as research experiment.
    \end{itemize}
  }
  \ifshowalgo
  \only<2>{\scriptsize
    \begin{algorithm}[H]
      \DontPrintSemicolon % Some LaTeX compilers require you to use
      \KwIn{
  $workerSet=\{w_1, w_2, \ldots, w_m\}$,
  $jobSet=\{j_1, j_2, \ldots, j_n\}$,
  preserving rate $r\in[0,1]$
}
\KwOut{A mapping of each job to scheduled workers}
$result \gets$
KeyValueMap(key $\to \{\}$ for all key $\in jobSet$)\;
$jobSet \gets$ $sortByPriority(jobSet)$\;
$c \gets$ $min(jobSet[0]$.totalTask, $workerSet$.size$-floor((1-r) *
workerSet.$size$)$\;
$result[jobSet[0]] \gets$ $\{w_1, ..., w_c\}$\;
\For{$i=1$ to $min(m-c, n)$}{
  $result[jobSet[i]]$ $\gets$ $\{w_{c+i}\}$\;
}
\Return{result}\;


      \caption{Priority-based policy}
      \label{algo:priority-based}
    \end{algorithm}
  }
  \fi
\end{frame}

\subsection{Proportion-based}
\begin{frame}
  \frametitle{Policy -- Proportion-based}
  %For streaming jobs, workloads may varies: For streaming jobs dependent on
  %others, if they don't have anything to process, its workload is low
  %so no workers needed, but as results come out, we should give it
  %servers
  \only<1>{
    \begin{itemize}
      \item Allocate resource proportionally according to workload 
      \item "Fair-share scheduling"~\footnote[frame]{\fullcite{cite:fair-share-scheduling}}
      \item Good for streaming jobs or jobs with dependency
    \end{itemize}
  }
  \ifshowalgo
  \only<2>{\scriptsize
    \begin{algorithm}[H]
      \DontPrintSemicolon % Some LaTeX compilers require you to use
      \KwIn{
  $workerSet=\{w_1, w_2, \ldots, w_m\}$,
  $jobSet=\{j_1, j_2, \ldots, j_n\}$,
}
\KwOut{A mapping of each job to scheduled workers}
$result \gets$ KeyValueMap(key $\to \{\}$ for all key $\in$
$jobSet$)\;
$jobSet \gets$ $sortByPriority(jobSet)$\;
\For{$i=1$ to $m$}{
  break if $workerSet$.size equals to 0 \;
  $c \gets$
  $min(ceil(\frac{jobSet[i].\text{workload}}{\text{total
  workload}})), workerSet.\text{size})$ \;
  $result[jobSet[i]] \gets \{\text{first $c$ element of
  $workerSet$}\}$\;
  $workerSet \setminus result[jobSet[i]]$\;
}
\Return{result}\;

      \caption{Proportion-based policy}
      \label{algo:proportion-based}
    \end{algorithm}
  }
  \fi
\end{frame}

\subsection{Workload-based}
\begin{frame}
  \frametitle{Policy -- Workload-based}
  \only<1>{
    \begin{itemize}
      \item Consider both priority and workload
      \item Give just enough resource to meet the requirement 
      \item Achieve \emph{heterogeneity aware}allocation by always allocating
        workers that run the job the fastest
    \end{itemize}
  }
  \ifshowalgo
  \only<2>{\scriptsize
    \begin{algorithm}[H]
      \DontPrintSemicolon % Some LaTeX compilers require you to use
      \KwIn{
  $workerSet=\{w_1, w_2, \ldots, w_m\}$,
  $jobSet=\{j_1, j_2, \ldots, j_n\}$,
}
\KwOut{A mapping of each job to scheduled workers}
$result \gets$
KeyValueMap(key $\to \{\}$ for all key $\in jobSet$)\;
$jobSet \gets$ $sortByPriority(jobSet)$\;
\For{each job $j_i=(w_i,p_i,n_i)  \in jobSet$}{
  \If{$workerSet$ is empty}{\textbf{break}\;}
  $cmp \gets function(w)$\{\Return $w.th[j_i]\}$\;
  $workerSet \gets$ $sort(workerSet,cmp,DECSENDING)$\;
  $th \gets 0$\;
  \For{\textbf{(} $k = 0$ ;\\
    $k<workerSet$.size \textbf{AND}
    $k<n_i$ \textbf{AND}
    $th < w_i$;
    $k \gets k+1 $\textbf{)}
  }{
    $th \mathrel{+}= workerSet[k].th[j_i]$\;
  }
  $result[j_i] \gets workerSet[0...i]$\;
  $workerSet$.remove(0,$k$)
}
assign remaining workers to jobs still in need\;
\Return{result}\;

      \caption{Workload-based policy}
      \label{algo:workload-based}
    \end{algorithm}
  }
  \fi
\end{frame}

\subsection{Deadline-based}

\begin{frame}
  \frametitle{Policy -- Deadline-based}
  %TODO for different needs, use different policies
  %TODO If we have workload and deadline
  %TODO Tell this is the main focus of CHT's request, so our experiments
  %focus on this
  \only<1>{
    \begin{itemize}
      \item Use deadlines to estimated required workload and apply
        the workload-based policy
      \item Allocate resource just enough to meet deadlines
      \item \emph{Heterogeneity} aware scheduling
      \item Scenario: billing accounting
    \end{itemize}
  }
  \ifshowalgo
  \only<2>{\scriptsize
    \begin{algorithm}[H]
      \DontPrintSemicolon % Some LaTeX compilers require you to use
      \KwIn{
  $workerSet=\{w_1, w_2, \ldots, w_m\}$,
  $jobSet=\{j_1, j_2, \ldots, j_n\}$,
}
\KwOut{A mapping of each job to scheduled workers}
% TODO Implement the algorithm

\Return{result}\;

      \caption{Deadline-based policy}
      \label{algo:deadline-based}
    \end{algorithm}
  }
  \fi
\end{frame}



\section{Q3 Progress}
\begin{frame}<beamer>
  \frametitle{Outline}
  \tableofcontents[
    currentsubsection,
    sectionstyle=show/shaded,
    subsectionstyle=show/show/shaded,
  ]
\end{frame}

\begin{frame}
  \frametitle{Q3 Progress}
  \begin{itemize}
    \item Task queuing
      \begin{itemize}
        \item To release trailing idle workers earlier
        \item Reduce resource waste
      \end{itemize}
    \item Adaptive adjustment
      \begin{itemize}
        \item Monitor job execution progress to estimate running time
      \end{itemize}
  \end{itemize}
\end{frame}

\subsection{Task Queuing}
\begin{frame}
  \frametitle{Task Queuing}
  \begin{itemize}
    \item Workers are occupied until all tasks of the assigned job are
      done
    \item Trailing idle workers
    \item Task queuing enables releasing trailing idle workers earlier
  \end{itemize}
\end{frame}

\begin{frame}
  \frametitle{Trailing Idle Worker}

  \begin{figure}[h]
    \centering
    \pgfdeclarelayer{background}
\pgfsetlayers{background,main}
\begin{tikzpicture}
  \matrix (wmtrx2)[worker-matrix]{
    \node()[fill=red!15]{$w_0$};& \node(i1)[invisible]{};& \node()[fill=red!15]{$w_0$};&	\node(j1)[invisible]{};&	\node()[fill=red!15]{$w_0$};\\
    \node()[fill=red!15]{$w_1$};& \node(i2)[invisible]{};& \node()[fill=red!15]{$w_1$};&	\node(j2)[invisible]{};&	\node()[fill=red!15]{$w_1$};\\
    \node()[fill=red!15]{$w_2$};& \node(i3)[invisible]{};& \node()[fill=red!15]{$w_2$};&	\node(j3)[invisible]{};&	\node(n2){$w_2$};\\
    \node()[fill=red!15]{$w_3$};& \node(i4)[invisible]{};& \node()[fill=red!15]{$w_3$};&	\node(j4)[invisible]{};&	\node(n3){$w_3$};\\
  };
  \path ($(i2)!0.5!(i3)$) node[fill=blue!30,isosceles triangle,isosceles triangle stretches=true,minimum height=4em,minimum width=10em]{};
  \path ($(j2)!0.5!(j3)$) node[fill=blue!30,isosceles triangle,isosceles triangle stretches=true,minimum height=4em,minimum width=10em]{};
  \begin{pgfonlayer}{background}
    \node[fit=(n2) (n3),draw=red,rectangle,very thick,rounded corners,inner sep=0.2cm,label=below:Trailing Idle Worker]{};
  \end{pgfonlayer}
\end{tikzpicture}

    \caption{Trailing Idle Worker}
  \end{figure}

\end{frame}

\subsection{Adaptive Adjustment}
\begin{frame}
  \frametitle{Adaptive Adjustment}
  \setbeamercovered{transparent}
  \only<1-5>{
    \begin{itemize}[<+->]
      \item We assumed that the execution time of each task can be
        predicted or obtained by profiling.
      \item <.->But accurate information is not always available.
        \begin{itemize}
          \item New applications
          \item Profiling error
          \item Machine performance degradation
        \end{itemize}
      \item Log the execution progress of each task and use it to
        estimate task execution speed
    \end{itemize}
  }
  \only<6>{
    \includegraphics[width=\textwidth]{figures/adaptive.png}
  }
\end{frame}

\section{Q4 TODOs}
\begin{frame}
  \frametitle{Integration with JPPF}
  \begin{itemize}[<+->]
      \only<1-2>{
      \item JPPF (Java Parallel Processing Framework) is a very popular open-source cluster management
        framework
        \begin{itemize}
          \item <.->Very easy to deploy
          \item <.->GUI monitoring tools
          \item <.->Active development
        \end{itemize}
      \item Does not support \emph{centralized} and
        \emph{node-aware} scheduling
      }
      \only<3->{
      \item It provides API for jobs to set "filters" for a job -- to reject a
        node of from running that job
      \item Leveraging this API, we might be able to implement node-aware
        scheduling by \pause
        \begin{enumerate}
          \item Contact the system for scheduling information
          \item Get scheduled worker of the job
          \item Reject if this node is not scheduled
        \end{enumerate}
      }
  \end{itemize}
\end{frame}

\end{document}
