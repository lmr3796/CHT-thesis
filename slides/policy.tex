\section{Scheduling Policies}

\begin{frame}
  \frametitle{Policy}
  \begin{itemize}
    \item Basically, the scheduling algorithm used by the system
    \item Each of them requires different parameters
      \begin{itemize}
        \item User provided
        \item System statistics
      \end{itemize}
    \item Administrators can select different policies for their need
  \end{itemize}
\end{frame}

\begin{frame}
  \frametitle{Policy -- Priority-based}
  %TODO for different needs, use different policies
  %If no deadline, only priority is specified. e.g., research experiment
  \only<1>{
    \begin{itemize}
      \item Make the job with highest priority done as fast as possible
      \item Preserve some workers to relieve starvation
      \item Only priority is considered
      \item Good for those without hard time limit situations like research experiment
    \end{itemize}
  }
  \only<2>{\scriptsize
    \begin{algorithm}[H]
      \DontPrintSemicolon % Some LaTeX compilers require you to use
      \KwIn{
  $workerSet=\{w_1, w_2, \ldots, w_m\}$,
  $jobSet=\{j_1, j_2, \ldots, j_n\}$,
  preserving rate $r\in[0,1]$
}
\KwOut{A mapping of each job to scheduled workers}
$result \gets$
KeyValueMap(key $\to \{\}$ for all key $\in jobSet$)\;
$jobSet \gets$ $sortByPriority(jobSet)$\;
$c \gets$ $min(jobSet[0]$.totalTask, $workerSet$.size$-floor((1-r) *
workerSet.$size$)$\;
$result[jobSet[0]] \gets$ $\{w_1, ..., w_c\}$\;
\For{$i=1$ to $min(m-c, n)$}{
  $result[jobSet[i]]$ $\gets$ $\{w_{c+i}\}$\;
}
\Return{result}\;


      \caption{Priority-based policy}
      \label{algo:priority-based}
    \end{algorithm}
  }
\end{frame}

\begin{frame}
  \frametitle{Policy -- Proportion-based}
  %For streaming jobs, workloads may varies: For streaming jobs dependent on
  %others, if they don't have anything to process, its workload is low
  %so no workers needed, but as results come out, we should give it
  %servers
  \only<1>{
    \begin{itemize}
      \item Take workload as main concern
      \item "Fair sharing"
      \item Considers only workload
      \item Good for streaming jobs with dependency
    \end{itemize}
  }
  \only<2>{\scriptsize
    \begin{algorithm}[H]
      \DontPrintSemicolon % Some LaTeX compilers require you to use
      \KwIn{
  $workerSet=\{w_1, w_2, \ldots, w_m\}$,
  $jobSet=\{j_1, j_2, \ldots, j_n\}$,
}
\KwOut{A mapping of each job to scheduled workers}
$result \gets$ KeyValueMap(key $\to \{\}$ for all key $\in$
$jobSet$)\;
$jobSet \gets$ $sortByPriority(jobSet)$\;
\For{$i=1$ to $m$}{
  break if $workerSet$.size equals to 0 \;
  $c \gets$
  $min(ceil(\frac{jobSet[i].\text{workload}}{\text{total
  workload}})), workerSet.\text{size})$ \;
  $result[jobSet[i]] \gets \{\text{first $c$ element of
  $workerSet$}\}$\;
  $workerSet \setminus result[jobSet[i]]$\;
}
\Return{result}\;

      \caption{Proportion-based policy}
      \label{algo:proportion-based}
    \end{algorithm}
  }
\end{frame}

\begin{frame}
  \frametitle{Policy -- Workload-based}
  \only<1>{
    \begin{itemize}
      \item Combining previous ones, take priority and workload into
        consideration
      \item Greedy: meet high priority requirements first
      \item Give resource that just meets requirement 
      \item Schedule jobs on those fit the most -- node
        difference/\emph{heterogeneity} aware
    \end{itemize}
  }
  \only<2>{\scriptsize
    \begin{algorithm}[H]
      \DontPrintSemicolon % Some LaTeX compilers require you to use
      \KwIn{
  $workerSet=\{w_1, w_2, \ldots, w_m\}$,
  $jobSet=\{j_1, j_2, \ldots, j_n\}$,
}
\KwOut{A mapping of each job to scheduled workers}
$result \gets$
KeyValueMap(key $\to \{\}$ for all key $\in jobSet$)\;
$jobSet \gets$ $sortByPriority(jobSet)$\;
\For{each job $j \in jobSet$}{
  \If{$workerSet$ is empty}{\textbf{break}\;}
  $cmp \gets function(w)$\{\Return $w$.throughput$[j]\}$\;
  $workerSet \gets$ $sort(workerSet,cmp,ACSENDING)$\;
  $i \gets 0$\;
  $throughput \gets 0$\;
  %\For{$i = 0 $ to $workerSet$.size-1}{
  %  $throughput \mathrel{+}= workerSet[i].throughput[j]$\;
  %  \If{$throughput \geq j$.workload}{
  %    \textbf{break}\;
  %  }
  %}
  \For{$i = 0$ ;\\
    $i<workerSet$.size \textbf{AND}
    $throughput < j$.workload;
    $i\mathrel{+}\mathrel{+}$
  }{
    $throughput \mathrel{+}= workerSet[i].throughput[j]$\;
  }
  $result[j] \gets workerSet[0...i]$\;
  $workerSet$.remove(0,$i$)
}

\Return{result}\;

      \caption{Deadline-based policy}
      \label{algo:deadline-based}
    \end{algorithm}
  }
\end{frame}

\begin{frame}
  \frametitle{Policy -- Deadline-based}
  %TODO for different needs, use different policies
  %TODO If we have workload and deadline
  %TODO Tell this is the main focus of CHT's request, so our experiments
  %focus on this
  \only<1>{
    \begin{itemize}
      \item Deadline is the main focus
      \item Use deadlines to estimated required workload and apply
        the workload-based policy
      \item Don't over schedule resource: give resource just enough to
        meet deadline
      \item Schedule jobs on those fit the most -- node
        difference/\emph{heterogeneity} aware
      \item CHT's main concern
      \item e.g. billing accountant
    \end{itemize}
  }
  \only<2>{\scriptsize
    \begin{algorithm}[H]
      \DontPrintSemicolon % Some LaTeX compilers require you to use
      \KwIn{
  $workerSet=\{w_1, w_2, \ldots, w_m\}$,
  $jobSet=\{j_1, j_2, \ldots, j_n\}$,
}
\KwOut{A mapping of each job to scheduled workers}
$result \gets$
KeyValueMap(key $\to \{\}$ for all key $\in jobSet$)\;
$jobSet \gets$ $sortByPriority(jobSet)$\;
\For{each job $j_i=(d_i,p_i,n_i)  \in jobSet$}{
  \If{$workerSet$ is empty}{\textbf{break}\;}
  $cmp \gets function(w)$\{\Return $1/w.execTime[j_i]\}$\;
  $workerSet \gets$ $sort(workerSet,cmp,DECSENDING)$\;
  $th \gets 0$\;
  \For{\textbf{(} $k = 0$ ;\\
    $k<workerSet$.size \textbf{AND}
    $k<n_i$ \textbf{AND}
    $th < 1/w_i.execTime[j_i]$;
    $k \gets k+1 $\textbf{)}
  }{
    $th \mathrel{+}= 1/workerSet[k].execTime[j_i]$\;
  }
  $result[j_i] \gets workerSet[0...i]$\;
  $workerSet$.remove(0,$k$)
}
assign remaining workers to jobs still in need\;
\Return{result}\;

      \caption{Deadline-based policy}
      \label{algo:deadline-based}
    \end{algorithm}
  }
\end{frame}

