\section{Implementation}
\subsection{RPC Server Design}
\begin{frame}
  \frametitle{RPC Server Design}
  We implement all the components in our system as RPC servers.
  \begin{itemize}
    \item Lowers coupling between components.
    \item Enables changing configurations or upgrading the system
      without halting the system.
    \item Roystonea\footnote[frame]{\tiny\fullcite{cite:roystonea}}
      benefits from this design, too.
  \end{itemize}
\end{frame}
\subsection{Task Queuing}
\begin{frame}
  \frametitle{Task Queuing}
  \begin{itemize}
    \item Workers are occupied until all tasks of the assigned job are
      done.
    \item Trailing idle workers.
    \item Maintain a available worker queue for each job.
    \item Enables releasing trailing idle workers earlier.
  \end{itemize}
\end{frame}

\begin{frame}
  \frametitle{Trailing Idle Worker}

  \begin{figure}[h]
    \centering
    \resizebox{\textheight}{!}{
      \pgfdeclarelayer{background}
\pgfsetlayers{background,main}
\begin{tikzpicture}
  \matrix (wmtrx2)[worker-matrix]{
    \node()[fill=red!15]{$w_0$};& \node(i1)[invisible]{};& \node()[fill=red!15]{$w_0$};&	\node(j1)[invisible]{};&	\node()[fill=red!15]{$w_0$};\\
    \node()[fill=red!15]{$w_1$};& \node(i2)[invisible]{};& \node()[fill=red!15]{$w_1$};&	\node(j2)[invisible]{};&	\node()[fill=red!15]{$w_1$};\\
    \node()[fill=red!15]{$w_2$};& \node(i3)[invisible]{};& \node()[fill=red!15]{$w_2$};&	\node(j3)[invisible]{};&	\node(n2){$w_2$};\\
    \node()[fill=red!15]{$w_3$};& \node(i4)[invisible]{};& \node()[fill=red!15]{$w_3$};&	\node(j4)[invisible]{};&	\node(n3){$w_3$};\\
  };
  \path ($(i2)!0.5!(i3)$) node[fill=blue!30,isosceles triangle,isosceles triangle stretches=true,minimum height=4em,minimum width=10em]{};
  \path ($(j2)!0.5!(j3)$) node[fill=blue!30,isosceles triangle,isosceles triangle stretches=true,minimum height=4em,minimum width=10em]{};
  \begin{pgfonlayer}{background}
    \node[fit=(n2) (n3),draw=red,rectangle,very thick,rounded corners,inner sep=0.2cm,label=below:Trailing Idle Worker]{};
  \end{pgfonlayer}
\end{tikzpicture}

    }
    \caption{Trailing Idle Worker}
  \end{figure}
\end{frame}

\subsection{Adaptive Adjustment}
\begin{frame}
  \frametitle{Adaptive Adjustment}
  \setbeamercovered{transparent}
  \only<1-5>{
    \begin{itemize}[<+->]
      \item We assumed that the execution time of each task can be
        predicted or obtained by profiling.
      \item <.->But accurate information is not always available.
        \begin{itemize}
          \item New applications
          \item Profiling error
          \item Machine performance degradation
        \end{itemize}
      \item Log the execution progress of each task and use it to
        estimate task execution speed
    \end{itemize}
  }
  \only<6>{
    \includegraphics[width=\textwidth]{figures/adaptive.png}
  }
\end{frame}
\begin{frame}
  \frametitle{Adaptive Adjustment (Cont'd)}
  \setbeamercovered{transparent}
  \begin{itemize}
    \item Improves fault tolerance.
      \begin{itemize}
        \item Detects worker failure and assign failed task to another
          worker.
      \end{itemize}
    \item Implies we can scheduling without considering reserving resource.
      \begin{itemize}
        \item When high priority jobs arrive, kill some lower priority
          ones and gives the resource to high priority ones.  Killed
          ones we be automatically redo when there is available
          resource.
        \item In our experiment, gives about \alert{3x} throughput.
      \end{itemize}
  \end{itemize}
\end{frame}

%\subsection{Case Study: Integration with JPPF}
%\begin{frame}
%  \frametitle{Case Study: Integration with JPPF}
%  \only<1>{
%    \begin{itemize}
%      \item JPPF\footnote[frame]{\tiny\fullcite{cite:JPPF}} is a very
%        popular open-source cluster management framework.
%        \begin{itemize}
%          \item Very easy to deploy
%          \item GUI monitoring tools
%          \item Active development
%        \end{itemize}
%      \item Used in CHT's cluster.
%      \item Doesn't support \emph{centralized} and
%        \emph{node-aware} scheduling
%    \end{itemize}
%  }
%  \only<2->{
%    \begin{itemize}[<+->]
%      \item Provides API for jobs set "filters" for a job --- to reject a
%        node from running that job
%      \item<.-> Leveraging this API, we can somehow implement node-aware
%        scheduling by
%        \pause
%        \begin{enumerate}
%          \item Contact the system for scheduling information
%          \item Get scheduled worker of the job
%          \item Reject if this node is not scheduled
%        \end{enumerate}
%    \end{itemize}
%  }
%\end{frame}
