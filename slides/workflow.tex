\section{Work Flow}
\begin{frame}
  \frametitle{Work Flow -- Job Submission}
  \begin{enumerate}
    \item Client submits a job batch to dispatcher.
    \item Dispatcher passes information to decision maker for a new
      schedule.
    \item Decision maker obtains required information from status
      checker and schedules.
    \item Dispatcher updates job-worker relationship table according to
      the new schedule.
  \end{enumerate}
\end{frame}
\begin{frame}
  \frametitle{Work Flow -- Task Execution}
  \begin{enumerate}
    \item Client forks a task thread and notifies dispatcher a task is
      ready to run.
    \item The task thread waits for an available worker. This is a
      blocking operation.
    \item After the task thread obtains a worker, client forks another
      task thread and goes 2.
    \item The first client thread contacts the assigned worker and sends
      required information about the task: command, parameters, etc.
    \item Client thread tells the worker to run the task and waits for
      the result. This is also a blocking call.
    \item After obtaining the result, client remove the task from the
      remaining task list and notifies the dispatcher to release the
      worker.
    \item Dispatcher assigns the released worker to next job
      according to the current schedule plan.
  \end{enumerate}
\end{frame}
\begin{frame}
  \frametitle{Work Flow -- Job Completion}
  \begin{enumerate}
    \item After the last task of a job finished, notifies dispatcher to
      remove a job.
    \item Dispatcher removes the job entry and reschedules.
  \end{enumerate}
\end{frame}
\begin{frame}
  \frametitle{Work Flow -- Monitoring}
  \begin{enumerate}
    \item Status checker periodically pings each worker for its status.
    \item Also on each ping, collects execution statistics.
    \item If a worker stuck at OCCUPIED status for too long, forcibly
      releases it.
    \item If it found servers fail or recover from failure, tells the
      dispatcher to update the schedule.
  \end{enumerate}
\end{frame}
\begin{frame}
  \frametitle{Rescheduling Conditions}
  \begin{itemize}
    \item Job change
    \item Worker change
  \end{itemize}
\end{frame}
