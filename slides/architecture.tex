\section{System Architecture}

\begin{frame}
  \frametitle{System Architecture -- Overview}
  \begin{columns}
    \begin{column}{.5\textwidth}
      \begin{figure}
        \resizebox{\linewidth}{!}{
          \pgfdeclarelayer{background}
\pgfdeclarelayer{foreground}
\pgfsetlayers{background,main,foreground}

\begin{tikzpicture}
  \node[](management){\pgfdeclarelayer{background}
\begin{tikzpicture}
  \node[](key-comp){\begin{tikzpicture}
  % Define distances for bordering
  \def\blockdist{2.3}
  \def\edgedist{2.5}

  % Draw components
  \node[component](status-checker){Status Checker};
  \path (status-checker.south)+(0,-4.2) node[component](decision-maker){Decision Maker};
  \path (status-checker.north east)+(\blockdist,0) node[dispatcher, anchor=north west](dispatcher){Dispatcher};
  \draw[vecArrow] ([xshift=-10]status-checker.south) to ([xshift=-10]decision-maker.north);
  \draw[vecArrow] ([xshift=10]decision-maker.north) to ([xshift=10]status-checker.south);

  % Draw interaction arrows
  \def\composhift{50}
  \draw[vecArrow] ([yshift=-10]status-checker.east) to ([yshift=-10+\composhift]dispatcher.west);
  \draw[vecArrow] ([yshift=10+\composhift]dispatcher.west) to([yshift=10]status-checker.east);

  \draw[vecArrow] ([yshift=-10]decision-maker.east) to ([yshift=-10-\composhift]dispatcher.west);
  \draw[vecArrow] ([yshift=10-\composhift]dispatcher.west) to ([yshift=10]decision-maker.east) ;
\end{tikzpicture}
};
  \path (key-comp.south) + (0, -0.5) node(comp-caption){Management System};
  \begin{pgfonlayer}{background}
    \node[system,fit=(key-comp) (comp-caption)]{} ;
  \end{pgfonlayer}
\end{tikzpicture}
};

  % Client
  %\coordinate (client-mid) at (management.east) + (5em,0);
  %\path (client-mid) + (0,-2) node[component,text width=5em]{Client};
  %\path (client-mid) + (0, 2) node[component,text width=5em]{Client};

  % Workers
  \coordinate (l1) at (management.south west);
  \coordinate (l2) at (management.south east);
  \path (l1) + (0, -3)
  node[component,text width=4em,minimum height=3em](w1){worker};
  \path (l2) + (0, -3)
  node[component,text width=4em,minimum height=3em](w4){worker};
  \path ($(w1)!0.33!(w4)$)
  node[component,text width=4em,minimum height=3em](w2){worker};
  \path ($(w1)!0.66!(w4)$)
  node[component,text width=4em,minimum height=3em](w3){worker};
  \draw[vecArrow] ([xshift=-5]$(l1)!0.2!(l2)$) to ([xshift=-5,yshift=3]w1.north);
  \draw[vecArrow] ([xshift=-5]$(l1)!0.4!(l2)$) to ([xshift=-5,yshift=3]w2.north);
  \draw[vecArrow] ([xshift=-5]$(l1)!0.6!(l2)$) to ([xshift=-5,yshift=3]w3.north);
  \draw[vecArrow] ([xshift=-5]$(l1)!0.8!(l2)$) to ([xshift=-5,yshift=3]w4.north);
  \draw[vecArrow] ([xshift=5,yshift=3]w1.north) to ([xshift=5]$(l1)!0.2!(l2)$);
  \draw[vecArrow] ([xshift=5,yshift=3]w2.north) to ([xshift=5]$(l1)!0.4!(l2)$);
  \draw[vecArrow] ([xshift=5,yshift=3]w3.north) to ([xshift=5]$(l1)!0.6!(l2)$);
  \draw[vecArrow] ([xshift=5,yshift=3]w4.north) to ([xshift=5]$(l1)!0.8!(l2)$);
\end{tikzpicture}


        }
        \caption{Architecture Overview}
        \label{fig:archi-overview}
      \end{figure}
    \end{column}
    \begin{column}{.5\textwidth}
      \begin{itemize}
        \item Each component is a RPC server
        \item As a cluster management system or extension
          components of existent cloud systems
      \end{itemize}
    \end{column}
  \end{columns}
\end{frame}

\begin{frame}
  \frametitle{System Architecture -- Client}
  \only<1>{
    \begin{itemize}
      \item The programming interface users sends jobs to the system for execution
      \item Users can specify attributes for jobs
        \begin{itemize}
          \item Deadline
          \item Priority
          \item \ldots
        \end{itemize}
      \item Supports background execution

    \end{itemize}
  }
  \only<2>{
    \newfloat{Example Code}{H}{myc}
    \begin{Example Code}
      \lstset{
        frame=LRBT,
        basicstyle=\ttfamily\tiny\color{black},
        commentstyle = \ttfamily\color{blue!50},
        keywordstyle=\ttfamily\color{green},
        stringstyle=\color{red!80},
      }
      \begin{lstlisting}[language=Ruby]
  client = Client.new
  client.register(SERVER_ADDRESS)
  client.start
  j1 = Job.new('Job1')
  j1.add_task Task.new(...)
  ... # Add more tasks
  j2 = Job.new('Job2')
  j2.add_task Task.new(...)
  ... # Add more tasks
  # 200-second deadline
  j1.deadline = j2.deadline = Time.now + 200.0
  # Submit j1 and j2 together
  # Background execution
  j12_waiter = client.submit_job([j1,j2])
  # Do other time consuming computation
  ...
  j3 = Job.new('Job3')
  j3.add_task Task.new(...)
  ... # Add more tasks
  # Remaining part can't run until j3 is done
  j3_waiter = client.submit_job(j3)
  client.wait(j3_waiter)
  # Some more things to do
  ...
  # Wait until j1 and j2 is done.
  client.wait(j12_waiter)
  # Combning j1, j2 and j3
  ...
\end{lstlisting}

      \caption{Sample code of client usage}
    \end{Example Code}
  }
\end{frame}

\begin{frame}
  \frametitle{System Architecture -- Worker}
  % TODO Hurray!!!!!! I like this part!!!!!
\end{frame}

\begin{frame}
  \frametitle{System Architecture -- Decision Maker}
  % TODO Hurray!!!!!! I like this part!!!!!
\end{frame}

\begin{frame}
  \frametitle{System Architecture -- Status Checker}
  % TODO Hurray!!!!!! I like this part!!!!!
\end{frame}

\begin{frame}
  \frametitle{System Architecture -- Dispatcher}
  % TODO Hurray!!!!!! I like this part!!!!!
\end{frame}

\begin{frame}
  \frametitle{Work flow -- Standalone Mode}
\end{frame}

\begin{frame}
  \frametitle{Work flow -- Case study: Integration with JPPF}
  \begin{itemize}[<+->]
      \only<1-2>{
      \item JPPF (Java Parallel Processing Framework) is a very popular open-source cluster management
        framework
        \begin{itemize}
          \item <.->Very easy to deploy
          \item <.->GUI monitoring tools
          \item <.->Active development
        \end{itemize}
      \item Doesn't support \emph{centralized} and
        \emph{node-aware} scheduling
      }
      \only<3->{
      \item Provides API for jobs set "filters" for a job -- to reject a
        node of from running that job
      \item Leveraging this API, we can somehow implement node-aware
        scheduling by \pause
        \begin{enumerate}
          \item Contact the system for scheduling information
          \item Get scheduled worker of the job
          \item Reject if this node is not scheduled
        \end{enumerate}
      }
  \end{itemize}
\end{frame}

