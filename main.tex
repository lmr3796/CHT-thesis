\documentclass[11pt]{article}

\title{Job Dispatching and Scheduling Algorithms under Heterogeneous
  Clusters}

\author{Ting-Chou Lin}
\date{}
\usepackage{url}

\begin{document}

\maketitle

\section{Introduction}

Scalability is one of the key characteristics of cloud computing.
% TODO: Add example why this is the key???? Why is it important?

Virtualization wraps essential software components into virtual machines
in order to make cloud systems scalable.  For example, Lombardi and Di
Pietro propose ACPS
%(add the name here)
that wraps applications, operating systems and libraries into virtual
machines, and deploy them on physical servers for
execution~\cite{secure_virt_for_cloud, cloud_issue}.  After we wrap
essential software components into virtual machines, we can deploy them
dynamically.  Consequently we can quickly adjust the number of running
virtual machines to satisfy user demands.

Properly placing multiple virtual machines on a single physical machine
will utilize its hardware resource more efficiently.  As a hypervisor
deploys more virtual machines on a physical machine, and allocates
resources for them, it improves system utilization by reducing the
possibility of resource sitting idle.  However, as the number of virtual
machines in a physical machine increases, the resource each virtual
machine receives will decrease~\cite{resource_overbooking}.  For
example, if we deploy a single virtual machines on a physical machine,
it can utilize all the physical resources.  However, if we deploy four
virtual machines on the same physical machine, each of them will receive
a quarter of the hardware resource, which makes applications running on
these virtual machines perform worse.

Run applications on dedicated servers can guarantee performance.  A
dedicated server~\cite{dedicated_hosting} runs at most one virtual
machine at a time, so that the performance of the virtual machine will
not be interfered by other virtual machines.
%TODO: More dedicate server and performance guarantee examples here.
%CHT exmaples here.

The concept of dedicated server is feasible in data centers.  The
number of physical servers of a data center is usually
fixed~\cite{maintenance_framework}, and it only changes infrequently
when we add newly purchased servers, or remove obsolete ones, during
equipment upgrade.  In other words, the number of physical servers
within a data center is fixed between upgrades.  Since the server
configuration is fixed, we can statically map applications to
dedicated servers.

Under the assumption that the number of physical servers in a data
center is fixed, determining how many dedicated servers should be
allocated for each job or application becomes a critical problem:

Matching application instances to dedicated servers is crucial to
performance.

Jobs with different priority requiring different processing throughput
will start and finish at different time; handling these varying
conditions and dynamically adjusting allocated resource of each job so
that every job could meet its requirement is what we want to do.

The goal of this project is to develop a cloud resource management
system that dynamically adjusts the number of dedicated server a
single job is granted to use according to a specified policy with
regard to the remaining workload, priority and deadline of the
job. The data center manager can choose a different policy to fit the
their need or specify a customized policy for the system; moreover,
this management system shall be able to cooperate with existing cloud
operating systems.

% There's a remaining paragraph in Simon's proposal which is added per
% CHT's request, but I think it's not appropriate to put it in the
% background section.

\section{Architecture}

% TODO: There's a figure of their relationship. Should redraw it with
% latex

The architecture of this cloud management system consists of three
components: the \emph{status checker}, the \emph{decision maker} and the
\emph{dispatcher}:  The \emph{status checker} periodically collects the
information about physical servers such as whether a node is down, busy
or idle and its resource usage.  Such information is provided to the
\emph{decision maker} as reference for making allocation adjustment.
The \emph{decision maker} is in charge of making and adjusting resource
allocation plans according to the specified policy that takes job
deadline, priority, etc. into consideration; it is designed to be in
\emph{passive mode}, which means it is invoked (by the
\emph{dispatcher}) only under certain circumstances (which will be
discussed later) rather than periodically.  The \emph{dispatcher} is the
component that deal with the physical resource allocation adjustment
according to the allocation plan made by the \emph{decision maker}.

\subsection{Status Checker}

\subsection{Decision Maker}

\subsection{Dispatcher}

\section{Policy}

%\renewcommand\refname{Reference}
\bibliographystyle{abbrev}
\bibliography{citation}

\end{document}
