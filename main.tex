\documentclass[11pt]{article}

\title{\textbf{Job Dispatching and Scheduling Algorithms under Heterogenous Clusters}}
\author{Ting-Chou Lin\\}
\date{}
\usepackage{url}

\begin{document}

\maketitle

\section{Introduction}

Scalability is one of the key characteristics of cloud computing.  In
order to make cloud systems scalable, we apply lots of virtualization
techniques~\cite{secure_virt_for_cloud, cloud_issue} to wrap
applications, operating systems and libraries into virtual machines
(VMs) and deploy them on physical servers for execution.  The number of
running virtual machines can be adjusted quickly according to user
requests.

Placing multiple virtual machines, which consumes resource of physical
machines according to the allocation made by the hypervisor, on a single
physical machine may utilize its hardware resource better.  However, as
the number of virtual machines within a single physical machine
increases, the resource each virtual machine get would probably
decrease~\cite{resource_overbooking}.  For example, if four identical
virtual machines with same priority are deployed on a single-core
physical machine, each of them would get only one-fourth of the CPU
time, which makes applications running on these virtual machines perform
worse then expected.

One way to retain undamaged performance for applications is to execute
them on dedicated servers~\cite{dedicated_hosting} -- servers that run
at most one virtual machine at a time respectively.  Different from
highly scalable virtual machines, the number of physical servers within
a data center is usually fixed~\cite{maintenance_framework}, except for
the addition with newly purchased server and the removal of obsolete
ones during equipment upgrade of a data center.  In other words, during
the period between equipment upgrade, we can treat the number of
physical servers within a data center as fixed.

Under the assumption that the number of physical servers in a data
center is fixed, determining how many dedicated servers should be
allocated for each job or application becomes a critical problem:  Jobs
with different priority requiring different processing throughput will
start and finish at different time; handling these varying conditions
and dynamically adjusting allocated resource of each job so that every
job could meet its requirement is what we want to do.

The goal of this project is to develop a cloud resource management
system that dynamically adjusts the number of dedicated server a single
job is granted to use according to a specified policy with regard to the
remaining workload, priority and deadline of the job. The data center
manager can choose a different policy to fit the their need or specify a
customized policy for the system; moreover, this management system shall
be able to cooperate with existing cloud operating systems.

% There's a remaining paragraph in Simon's proposal which is added per
% CHT's request, but I think it's not appropriate to put it in the
% background section.

\section{Architecture}

% TODO: There's a figure of their relationship. Should redraw it with
% latex

The architecture of this cloud management system consists of three
components: the \emph{status checker}, the \emph{decision maker} and the
\emph{dispatcher}:  The \emph{status checker} periodically collects the
information about physical servers such as whether a node is down, busy
or idle, its resource usage\ldots.  Such information is provided to the
\emph{decision maker} as reference for making allocation adjustment.
The \emph{decision maker} is in charge of making and adjusting resource
allocation plans according to the specified policy that takes job
deadline, priority, etc. into consideration; it is designed to be in
\emph{passive mode}, which means it is invoked (by the
\emph{dispatcher}) only under certain circumstances (which will be
discussed later) rather than periodically.  The \emph{dispatcher} is the
component that deal with the physical resource allocation adjustment
according to the allocation plan made by the \emph{decision maker}.

\subsection{Status Checker}

\subsection{Decision Maker}

\subsection{Dispatcher}

\section{Policy}

\renewcommand\refname{Reference}
\bibliographystyle{unsrt}
\bibliography{citation}

\end{document}
