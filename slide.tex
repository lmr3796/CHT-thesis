\RequirePackage{tikz}
\RequirePackage{url}
\documentclass{beamer}
\usetheme{Warsaw}

\usepackage[linesnumbered,ruled]{algorithm2e}
\usepackage{amsmath}
\usepackage{float}
\usepackage{listings}
\usepackage{url}
\usepackage{tikz}

\title{Job Dispatching and Scheduling under Heterogeneous Clusters}
\author{Ting-Chou Lin}

\begin{document}

\begin{frame}
  \titlepage
  \label{title-page}
\end{frame}
\begin{frame}
  \frametitle{Background}
  \setbeamercovered{transparent}
  % TODO Describe jobs
  \begin{itemize}[<+->]
    \item As the amount of data and computing demand increases, more and
      more company tend to build their own data centers -- Private cloud
    % Deploying all applications to the cluster
    \item For elasticity, rather than dedicating servers for particular
      applications, %TODO finish this!!!!!!
  \end{itemize}
\end{frame}
\begin{frame}
  \frametitle{Background -- Job Characteristics}
  \setbeamercovered{transparent}
  \begin{itemize}[<+->]
    \item In data centers, jobs come with different characteristics
      \begin{itemize}
        \item Importance/Priority
        \item Computation resource demand
          \begin{itemize}
            \item CPU $\leftrightarrow$ I/O bound
            \item <.-> Requires large RAM
            \item <.-> Leveraging GPGPU
          \end{itemize}
        \item <.(2)->{\alert<.(2)->{Deadline}}
      \end{itemize}
    \item Moreover, some of them should be completed before time
      constraints
      \begin{itemize}
        \item <.-> Billing jobs should be done within days after charge-off
        \item <.-> Research experiments are not that urgent
      \end{itemize}
  \end{itemize}
\end{frame}
\begin{frame}
  \frametitle{Background -- Server Characteristics}
  \setbeamercovered{transparent}
  \begin{itemize}[<+->]
    \item Resources in data centers are usually \emph{limited} and
      \emph{fixed}
      \begin{itemize}
        \item Purchase new ones and remove obsolete ones only on
          periodic equipment upgrades
        \item \textbf{We can view the number of workers as fixed between
          upgrades}
      \end{itemize}
    \item <.(-2)-> Servers in the cluster are \alert{heterogeneous}
      \begin{itemize}
        \item Old servers coexist with new servers
        \item <.-> Only few servers have special capabilities, like Giga-LAN
          or GPGPU
      \end{itemize}
  \end{itemize}
\end{frame}
\begin{frame}
  \frametitle{Goal}
  \begin{itemize}
    \item Implement a resource management component that \emph{dynamically}
      adjust resource for each job
    \item System administrator can easily specify different policy to
      adapt to their needs
    \item The system should take server heterogeneity into consideration
  \end{itemize}
\end{frame}
\begin{frame}
  \frametitle{Target Environment -- CHT's Data Center}
  \begin{itemize}
    \item Old and new servers coexists -- heterogeneous environment
    \item Jobs are mainly about processing files (billing accountant)
    \item Jobs are split into many \emph{tasks}, each of which can be processed
      in parallel
    \item Data are shared via NFS
    \item Servers may fail
  \end{itemize}
\end{frame}
\begin{frame}[label=my-frame]
  \frametitle{System Architecture -- Overview}
  % TODO use the figure
  % TODO add workers to the figure
  % TODO RPC implementation
  % TODO emphasize that it can work "standalone" or become a part of
  % other system
  \begin{columns}
    \begin{column}{.5\textwidth}
      \begin{figure}
        \hspace*{-5.1cm}
        \scalebox{0.5}{
          \usetikzlibrary[backgrounds]
\begin{tikzpicture}
  \node[](key-comp){\begin{tikzpicture}
  \usetikzlibrary{arrows, shadows, decorations.markings}
  \tikzstyle{vecArrow} = [thick, decoration={markings,mark=at position
	1 with {\arrow[semithick]{open triangle 60}}},
	double distance=1.4pt, shorten >= 5.5pt, preaction = {decorate},
	postaction = {draw,line width=2pt, white,shorten >= 4.5pt}
  ]
  \tikzstyle{component} = [draw,text centered,rounded corners,drop
  shadow,text width=8em,fill=red!20,minimum height=6em]
  \tikzstyle{dispatcher} = [component,minimum height=19.3em]

  % Define distances for bordering
  \def\blockdist{2.3}
  \def\edgedist{2.5}

  % Draw components
  \node[component](status-checker){Status Checker};
  \path (status-checker.north east)+(\blockdist,0) node[dispatcher, anchor=north west](dispatcher){Dispatcher};
  \path (status-checker.south)+(0,-4.2) node[component](decision-maker){Decision Maker};
  \draw[vecArrow] ([xshift=-10]status-checker.south) to ([xshift=-10]decision-maker.north);
  \draw[vecArrow] ([xshift=10]decision-maker.north) to ([xshift=10]status-checker.south);

  %Draw interaction arrows
  \def\composhift{50}
  \draw[vecArrow] ([yshift=-10]status-checker.east) to ([yshift=-10+\composhift]dispatcher.west);
  \draw[vecArrow] ([yshift=10+\composhift]dispatcher.west) to([yshift=10]status-checker.east);

  \draw[vecArrow] ([yshift=-10]decision-maker.east) to ([yshift=-10-\composhift]dispatcher.west);
  \draw[vecArrow] ([yshift=10-\composhift]dispatcher.west) to ([yshift=10]decision-maker.east) ;
\end{tikzpicture}
};
  \begin{pgfonlayer}{background}
  	%\node[text centered,minimum width=4em,rounded corners,fill=yellow!25]{};
	\node[draw,dashed,rounded corners,fill=yellow!30](management) {Management System} ;
	%(key-comp.north west) rectangle (key-comp.south east);
  \end{pgfonlayer}
\end{tikzpicture}


        }
        \caption{Architecture Overview}
        \label{fig:archi-overview}
      \end{figure}
    \end{column}
    \begin{column}{.5\textwidth}
      \begin{itemize}
        \item Each component is a RPC server
        \item As a cluster management system or extension
          components of existent cloud systems
      \end{itemize}
    \end{column}
  \end{columns}
\end{frame}
\begin{frame}
  \frametitle{System Architecture -- Client}
  % TODO Hurray!!!!!! I like this part!!!!!
\end{frame}
\begin{frame}
  \frametitle{System Architecture -- Decision Maker}
  % TODO Hurray!!!!!! I like this part!!!!!
\end{frame}
\begin{frame}
  \frametitle{System Architecture -- Status Checker}
  % TODO Hurray!!!!!! I like this part!!!!!
\end{frame}
\begin{frame}
  \frametitle{System Architecture -- Dispatcher}
  % TODO Hurray!!!!!! I like this part!!!!!
\end{frame}
\begin{frame}
  \frametitle{Work flow -- Standalone Mode}
\end{frame}
\begin{frame}
  \frametitle{Work flow -- Case study: Integration with JPPF}
\end{frame}
\begin{frame}
  \frametitle{Policy}
  %TODO for different needs, use different policies
\end{frame}
\begin{frame}
  \frametitle{Policy -- Priority-based}
  %TODO for different needs, use different policies
  %If no deadline, only priority is specified. e.g., research experiment
\end{frame}
\begin{frame}
  \frametitle{Policy -- Proportion-based}
  %TODO For streaming jobs, workloads may varies: For streaming jobs dependent on
  %others, if they don't have anything to process, its workload is low
  %so no workers needed, but as results come out, we should give it
  %servers
\end{frame}
\begin{frame}
  \frametitle{Policy -- Deadline-based}
  %TODO for different needs, use different policies
  %TODO If we have workload and deadline
  %TODO Tell this is the main focus of CHT's request, so our experiments
  %focus on this
\end{frame}
\begin{frame}
  \frametitle{Experiment}
  % TODO FUCKKKKKKKKKKKKKK!!!
\end{frame}

\end{document}
