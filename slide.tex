\documentclass{beamer}

% Put required libraries into library.tex
\usepackage[linesnumbered,ruled]{algorithm2e}
\usepackage{amsmath}
\usepackage{array}
\usepackage{float}
\usepackage{listings}
\usepackage{url}
\usepackage{setspace}
\usepackage{tikz}

\usetikzlibrary{arrows}
\usetikzlibrary[backgrounds]
\usetikzlibrary{calc}
\usetikzlibrary{decorations.markings}
\usetikzlibrary{fit}
\usetikzlibrary{shadows}
\usetikzlibrary{shapes.arrows}
\usetikzlibrary{shapes.geometric}



% Tikz figure styls
\tikzstyle{vecArrow} = [thick, decoration={markings,mark=at position
  1 with {\arrow[semithick]{open triangle 60}}},
  double distance=1.4pt, shorten >= 5.5pt, preaction = {decorate},
  postaction = {draw,line width=2pt, white,shorten >= 4.5pt}
]
\tikzstyle{component} = [draw,text centered,rounded corners,drop
  shadow,text width=8em,fill=red!20,minimum height=6em
]
\tikzstyle{dispatcher} = [component,minimum height=19.3em]
\tikzstyle{system}=[draw,dashed,rounded corners,fill=yellow!30]


\usetheme{Warsaw}

\title{Job Dispatching and Scheduling under Heterogeneous Clusters}
\author{Ting-Chou Lin}

\begin{document}

\begin{frame}
  \titlepage
  \label{title-page}
\end{frame}
\begin{frame}
  \frametitle{Background}
  \setbeamercovered{transparent}
  % TODO Describe jobs
  \begin{itemize}[<+->]
    \item As the amount of data and computing demand increases, more and
      more company tend to build their own data centers -- Private cloud
    % Deploying all applications to the cluster
    \item For elasticity, rather than dedicating servers for particular
      applications, %TODO finish this!!!!!!
  \end{itemize}
\end{frame}
\begin{frame}
  \frametitle{Background -- Job Characteristics}
  \setbeamercovered{transparent}
  \begin{itemize}[<+->]
    \item In data centers, jobs come with different characteristics
      \begin{itemize}
        \item Importance/Priority
        \item Computation resource demand
          \begin{itemize}
            \item CPU $\leftrightarrow$ I/O bound
            \item <.-> Requires large RAM
            \item <.-> Leveraging GPGPU
          \end{itemize}
        \item <.(2)->{\alert<.(2)->{Deadline}}
      \end{itemize}
    \item Moreover, some of them should be completed before time
      constraints
      \begin{itemize}
        \item <.-> Billing jobs should be done within days after charge-off
        \item <.-> Research experiments are not that urgent
      \end{itemize}
  \end{itemize}
\end{frame}
\begin{frame}
  \frametitle{Background -- Server Characteristics}
  \setbeamercovered{transparent}
  \begin{itemize}[<+->]
    \item Resources in data centers are usually \emph{limited} and
      \emph{fixed}
      \begin{itemize}
        \item Purchase new ones and remove obsolete ones only on
          periodic equipment upgrades
        \item \textbf{We can view the number of workers as fixed between
          upgrades}
      \end{itemize}
    \item <.(-2)-> Servers in the cluster are \alert{heterogeneous}
      \begin{itemize}
        \item Old servers coexist with new servers
        \item <.-> Only few servers have special capabilities, like Giga-LAN
          or GPGPU
      \end{itemize}
  \end{itemize}
\end{frame}
\begin{frame}
  \frametitle{Goal}
  \begin{itemize}
    \item Implement a resource management component that \emph{dynamically}
      adjust resource for each job
    \item System administrator can easily specify different policy to
      adapt to their needs
    \item The system should take server heterogeneity into consideration
  \end{itemize}
\end{frame}
\begin{frame}
  \frametitle{Target Environment -- CHT's Data Center}
  \begin{itemize}
    \item Old and new servers coexists -- heterogeneous environment
    \item Jobs are mainly about processing files (billing accountant)
    \item Jobs are split into many \emph{tasks}, each of which can be processed
      in parallel
    \item Data are shared via NFS
    \item Servers may fail
  \end{itemize}
\end{frame}
\begin{frame}
  \frametitle{System Architecture -- Overview}
  \begin{columns}
    \begin{column}{.5\textwidth}
      \begin{figure}
        \resizebox{\linewidth}{!}{
          \pgfdeclarelayer{background}
\pgfdeclarelayer{foreground}
\pgfsetlayers{background,main,foreground}

\begin{tikzpicture}
  \node[](management){\pgfdeclarelayer{background}
\begin{tikzpicture}
  \node[](key-comp){\begin{tikzpicture}
  % Define distances for bordering
  \def\blockdist{2.3}
  \def\edgedist{2.5}

  % Draw components
  \node[component](status-checker){Status Checker};
  \path (status-checker.south)+(0,-4.2) node[component](decision-maker){Decision Maker};
  \path (status-checker.north east)+(\blockdist,0) node[dispatcher, anchor=north west](dispatcher){Dispatcher};
  \draw[vecArrow] ([xshift=-10]status-checker.south) to ([xshift=-10]decision-maker.north);
  \draw[vecArrow] ([xshift=10]decision-maker.north) to ([xshift=10]status-checker.south);

  % Draw interaction arrows
  \def\composhift{50}
  \draw[vecArrow] ([yshift=-10]status-checker.east) to ([yshift=-10+\composhift]dispatcher.west);
  \draw[vecArrow] ([yshift=10+\composhift]dispatcher.west) to([yshift=10]status-checker.east);

  \draw[vecArrow] ([yshift=-10]decision-maker.east) to ([yshift=-10-\composhift]dispatcher.west);
  \draw[vecArrow] ([yshift=10-\composhift]dispatcher.west) to ([yshift=10]decision-maker.east) ;
\end{tikzpicture}
};
  \path (key-comp.south) + (0, -0.5) node(comp-caption){Management System};
  \begin{pgfonlayer}{background}
    \node[system,fit=(key-comp) (comp-caption)]{} ;
  \end{pgfonlayer}
\end{tikzpicture}
};

  % Client
  %\coordinate (client-mid) at (management.east) + (5em,0);
  %\path (client-mid) + (0,-2) node[component,text width=5em]{Client};
  %\path (client-mid) + (0, 2) node[component,text width=5em]{Client};

  % Workers
  \coordinate (l1) at (management.south west);
  \coordinate (l2) at (management.south east);
  \path (l1) + (0, -3)
  node[component,text width=4em,minimum height=3em](w1){worker};
  \path (l2) + (0, -3)
  node[component,text width=4em,minimum height=3em](w4){worker};
  \path ($(w1)!0.33!(w4)$)
  node[component,text width=4em,minimum height=3em](w2){worker};
  \path ($(w1)!0.66!(w4)$)
  node[component,text width=4em,minimum height=3em](w3){worker};
  \draw[vecArrow] ([xshift=-5]$(l1)!0.2!(l2)$) to ([xshift=-5,yshift=3]w1.north);
  \draw[vecArrow] ([xshift=-5]$(l1)!0.4!(l2)$) to ([xshift=-5,yshift=3]w2.north);
  \draw[vecArrow] ([xshift=-5]$(l1)!0.6!(l2)$) to ([xshift=-5,yshift=3]w3.north);
  \draw[vecArrow] ([xshift=-5]$(l1)!0.8!(l2)$) to ([xshift=-5,yshift=3]w4.north);
  \draw[vecArrow] ([xshift=5,yshift=3]w1.north) to ([xshift=5]$(l1)!0.2!(l2)$);
  \draw[vecArrow] ([xshift=5,yshift=3]w2.north) to ([xshift=5]$(l1)!0.4!(l2)$);
  \draw[vecArrow] ([xshift=5,yshift=3]w3.north) to ([xshift=5]$(l1)!0.6!(l2)$);
  \draw[vecArrow] ([xshift=5,yshift=3]w4.north) to ([xshift=5]$(l1)!0.8!(l2)$);
\end{tikzpicture}


        }
        \caption{Architecture Overview}
        \label{fig:archi-overview}
      \end{figure}
    \end{column}
    \begin{column}{.5\textwidth}
      \begin{itemize}
        \item Each component is a RPC server
        \item As a cluster management system or extension
          components of existent cloud systems
      \end{itemize}
    \end{column}
  \end{columns}
\end{frame}
\begin{frame}
  \frametitle{System Architecture -- Client}
  \only<1>{
    \begin{itemize}
      \item The programming interface users sends jobs to the system for execution
      \item Users can specify attributes for jobs
        \begin{itemize}
          \item Deadline
          \item Priority
          \item \ldots
        \end{itemize}
      \item Supports background execution

    \end{itemize}
  }
  \only<2>{
    \newfloat{Example Code}{H}{myc}
    \begin{Example Code}
      \lstset{
        frame=LRBT,
        basicstyle=\ttfamily\tiny\color{black},
        commentstyle = \ttfamily\color{blue!50},
        keywordstyle=\ttfamily\color{green},
        stringstyle=\color{red!80},
      }
      \begin{lstlisting}[language=Ruby]
  client = Client.new
  client.register(SERVER_ADDRESS)
  client.start
  j1 = Job.new('Job1')
  j1.add_task Task.new(...)
  ... # Add more tasks
  j2 = Job.new('Job2')
  j2.add_task Task.new(...)
  ... # Add more tasks
  # 200-second deadline
  j1.deadline = j2.deadline = Time.now + 200.0
  # Submit j1 and j2 together
  # Background execution
  j12_waiter = client.submit_job([j1,j2])
  # Do other time consuming computation
  ...
  j3 = Job.new('Job3')
  j3.add_task Task.new(...)
  ... # Add more tasks
  # Remaining part can't run until j3 is done
  j3_waiter = client.submit_job(j3)
  client.wait(j3_waiter)
  # Some more things to do
  ...
  # Wait until j1 and j2 is done.
  client.wait(j12_waiter)
  # Combning j1, j2 and j3
  ...
\end{lstlisting}

      \caption{Sample code of client usage}
    \end{Example Code}
  }
\end{frame}
\begin{frame}
  \frametitle{System Architecture -- Worker}
  % TODO Hurray!!!!!! I like this part!!!!!
\end{frame}
\begin{frame}
  \frametitle{System Architecture -- Decision Maker}
  % TODO Hurray!!!!!! I like this part!!!!!
\end{frame}
\begin{frame}
  \frametitle{System Architecture -- Status Checker}
  % TODO Hurray!!!!!! I like this part!!!!!
\end{frame}
\begin{frame}
  \frametitle{System Architecture -- Dispatcher}
  % TODO Hurray!!!!!! I like this part!!!!!
\end{frame}
\begin{frame}
  \frametitle{Work flow -- Standalone Mode}
\end{frame}
\begin{frame}
  \frametitle{Work flow -- Case study: Integration with JPPF}
  \begin{itemize}[<+->]
      \only<1-2>{
      \item JPPF (Java Parallel Processing Framework) is a very popular open-source cluster management
        framework
        \begin{itemize}
          \item <.->Very easy to deploy
          \item <.->GUI monitoring tools
          \item <.->Active development
        \end{itemize}
      \item Doesn't support \emph{centralized} and
        \emph{node-aware} scheduling
      }
      \only<3->{
      \item Provides API for jobs set "filters" for a job -- to reject a
        node of from running that job
      \item Leveraging this API, we can somehow implement node-aware
        scheduling by \pause
        \begin{enumerate}
          \item Contact the system for scheduling information
          \item Get scheduled worker of the job
          \item Reject if this node is not scheduled
        \end{enumerate}
      }
  \end{itemize}
\end{frame}
\begin{frame}
  \frametitle{Policy}
  \begin{itemize}
    \item Basically, the scheduling algorithm used by the system
    \item Each of them requires different parameters
      \begin{itemize}
        \item User provided
        \item System statistics
      \end{itemize}
    \item Administrators can select different policies for their need
  \end{itemize}
\end{frame}
\begin{frame}
  \frametitle{Policy -- Priority-based}
  %TODO for different needs, use different policies
  %If no deadline, only priority is specified. e.g., research experiment
  \only<1>{
    \begin{itemize}
      \item Make the job with highest priority done as fast as possbile
      \item Preserve some workers to relieve starvation
      \item Only priority is considered
      \item e.g. research experiment
    \end{itemize}
  }
  \only<2>{\scriptsize
    \begin{algorithm}[H]
      \DontPrintSemicolon % Some LaTeX compilers require you to use
      \KwIn{
  $workerSet=\{w_1, w_2, \ldots, w_m\}$,
  $jobSet=\{j_1, j_2, \ldots, j_n\}$,
  preserving rate $r\in[0,1]$
}
\KwOut{A mapping of each job to scheduled workers}
$result \gets$
KeyValueMap(key $\to \{\}$ for all key $\in jobSet$)\;
$jobSet \gets$ $sortByPriority(jobSet)$\;
$c \gets$ $min(jobSet[0]$.totalTask, $workerSet$.size$-floor((1-r) *
workerSet.$size$)$\;
$result[jobSet[0]] \gets$ $\{w_1, ..., w_c\}$\;
\For{$i=1$ to $min(m-c, n)$}{
  $result[jobSet[i]]$ $\gets$ $\{w_{c+i}\}$\;
}
\Return{result}\;


      \caption{Priority-based policy}
      \label{algo:priority-based}
    \end{algorithm}
  }
\end{frame}
\begin{frame}
  \frametitle{Policy -- Proportion-based}
  %For streaming jobs, workloads may varies: For streaming jobs dependent on
  %others, if they don't have anything to process, its workload is low
  %so no workers needed, but as results come out, we should give it
  %servers
  \only<1>{
    \begin{itemize}
      \item Take workload as main concern
      \item "Fair sharing"
      \item Considers only workload
      \item e.g. streaming jobs with dependency
    \end{itemize}
  }
  \only<2>{\scriptsize
    \begin{algorithm}[H]
      \DontPrintSemicolon % Some LaTeX compilers require you to use
      \KwIn{
  $workerSet=\{w_1, w_2, \ldots, w_m\}$,
  $jobSet=\{j_1, j_2, \ldots, j_n\}$,
}
\KwOut{A mapping of each job to scheduled workers}
$result \gets$ KeyValueMap(key $\to \{\}$ for all key $\in$
$jobSet$)\;
$jobSet \gets$ $sortByPriority(jobSet)$\;
\For{$i=1$ to $m$}{
  break if $workerSet$.size equals to 0 \;
  $c \gets$
  $min(ceil(\frac{jobSet[i].\text{workload}}{\text{total
  workload}})), workerSet.\text{size})$ \;
  $result[jobSet[i]] \gets \{\text{first $c$ element of
  $workerSet$}\}$\;
  $workerSet \setminus result[jobSet[i]]$\;
}
\Return{result}\;

      \caption{Proportion-based policy}
      \label{algo:proportion-based}
    \end{algorithm}
  }
\end{frame}
\begin{frame}
  \frametitle{Policy -- Deadline-based}
  %TODO for different needs, use different policies
  %TODO If we have workload and deadline
  %TODO Tell this is the main focus of CHT's request, so our experiments
  %focus on this
  \only<1>{
    \begin{itemize}
      \item Deadline is the main focus
      \item Takes priority, workload and deadline into consideration
      \item Greedy: meet high priority deadlines, give up low priority
        ones
      \item Don't over schedule resource: give resource just enough to
        make deadline
      \item Schedule jobs on those fit the most -- node difference
        (\emph{heterogeneity}) aware
      \item CHT's main concern
      \item e.g. billing accountant
    \end{itemize}
  }
  \only<2>{\scriptsize
      \begin{algorithm}[H]
      \DontPrintSemicolon % Some LaTeX compilers require you to use
      \KwIn{
  $workerSet=\{w_1, w_2, \ldots, w_m\}$,
  $jobSet=\{j_1, j_2, \ldots, j_n\}$,
}
\KwOut{A mapping of each job to scheduled workers}
% TODO Implement the algorithm

\Return{result}\;

      \caption{Deadline-based policy}
      \label{algo:deadline-based}
    \end{algorithm}
  }
\end{frame}
\begin{frame}
  \frametitle{Experiment}
  % TODO FUCKKKKKKKKKKKKKK!!!
\end{frame}

\end{document}
