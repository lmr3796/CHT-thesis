\section{Experiment}\label{sec:exp}

\subsection{Experimental Settings}

Our experimental environment consists of one master node and ten working nodes.
The master node is a physical machine with two quad-core CPU.
The CPU model is Intel(R) Xeon(R) CPU X5570 @ 2.93GHz.
On the other hand, each working node is a single-core physical machine.
The memory sizes are 1.5GB and 768MB, respectively.
Our management system works on the master node.
The master node also serves as a client which generate the workloads.
There are two workers on each working node.

%%% TODO Add citation of the workload
%%% TODO explain how we sample (and scale)
We conduct a trace-based simulation to demonstrate the efficiency of our system.
The trace we use is the {\em CERIT-SC workload log}, which is provided by the CERIT-SC and the Czech National Grid Infrastructure MetaCentrum.
The data sets, which contains 17,900 jobs, are generated from TORQUE traces during the first 3 months of the year 2013.
We take samples from these eighteen-thousand jobs, and scale the waiting and execution time of the sampled jobs.
The sample rate is XXX.

During simulation, the client starts new jobs according to the arrival time and execution time from the trace.
Since the actual workloads is not available, a worker will be set to ``sleep mode'' after receiving a task from the client.
The sleeping duration is equal to the task execution length.
After resumed from the sleeping mode, the worker sends a message to the management system indicating it has finished a task.


\subsection{Experimental Results}

%%% TODO ...= =+


